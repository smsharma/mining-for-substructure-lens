\documentclass[twocolumn]{aastex62}

\newcommand{\acronym}[1]{{\small{#1}}}
\newcommand{\package}[1]{\textsl{#1}}
\newcommand{\Euclid}{\textsl{Euclid}~}
\newcommand{\lcdm}{\acronym{$\Lambda$CDM}}
\newcommand{\kpc}{\textrm{kpc}}
\newcommand{\Msun}{\textrm{M}_\odot}

\shorttitle{Dark matter substructure with likelihood-free inference}
\shortauthors{Brehmer et al.}
%@arxiver{}

\begin{document}\sloppy\sloppypar\raggedbottom\frenchspacing

\title{\textbf{%
Mining for Substructure: \\
Inferring subhalo population properties in strong lenses with machine learning
}}

% \correspondingauthor{Siddharth Mishra-Sharma}
% \email{sm8383@nyu.edu}

\author{Johann Brehmer}
\affil{Center for Cosmology and Particle Physics, Department of Physics, New York University, 726~Broadway, New York, NY 10003, USA}
\affil{Center for Data Science, New York University, 60 Fifth Ave, New York, NY 10011, USA}

\author{Siddharth Mishra-Sharma}
\affil{Center for Cosmology and Particle Physics, Department of Physics, New York University, 726~Broadway, New York, NY 10003, USA}
\email{sm8383@nyu.edu}

\author{Joeri Hermans}
\affil{University of Li\`ege, Belgium}

\author{Gilles Louppe}
\affil{University of Li\`ege, Belgium}

\author{Kyle Cranmer}
\affil{Center for Cosmology and Particle Physics, Department of Physics, New York University, 726~Broadway, New York, NY 10003, USA}
\affil{Center for Data Science, New York University, 60 Fifth Ave, New York, NY 10011, USA}

\begin{abstract}\noindent
We develop methods to analyze a statistical sample of strong lenses in a principled way to look for dark matter substructure with likelihood-free inference techniques.
\end{abstract}

\keywords{%
strong lensing 
  ---
dark matter 
  --- 
cosmology
  ---
likelihood-free inference 
  --- 
machine learning
}

\section{Introduction}

\section{Strong lensing formalism}

\section{Simulating a population of lenses}

In strong lensing systems, the background light emission source can in general be a point-like quasar or supernova, or a faint, extended ``blue'' galaxy. The former results in multiple localized images on the lens plane rather than extended arc-like images, providing the ability to probe substructure over a limited region on the lens plane. For this reason, we focus our method towards galaxy-galaxy lense---systems producing images with extended arcs---since we aim to disentangle the collective, sub-threshold effect of a population of subhalo perturbers over multiple images. Young, blue galaxies are uniquitous in the redshift regime $z\gtrsim1$ and dominate the faint end of the galaxy luminosity function, resulting in a much larger deliverable sample of galaxy-galaxy strong lenses compared to quadruply- and doubly-imaged quasars/supernovae.

The fact that the strong lens population is expected to be dominated by higher-redshift ($z\gtrsim1$) blue source galaxies lensed by intermediate-redshift ($z\sim 0.5$--$1$) elliptical galaxies presents significant challenges for quantifying the lens population obtainable with future observations. Specifically, planned ground-based surveys like LSST and space telescopes like \Euclid present complementary challenges for delivering images of strong lensing systems suitable for substructure studies. LSST is expected to image in six bands, allowing efficient source selection and distinguishing source and lens emission, but at the cost of lower resolution by virtue of being a ground-based instrument. \emph{Euclid} imaging is expected be much higher in resolution but with a single optical passband (\emph{VIS}). Near-IR imaging from WFIRST may deliver a high-resolution, multi-wavelength dataset that is more suitable for substructure studies, although the lens and source populations may differ from those probed by optical telescopes. 

In light of these uncertainties, we limit the scope of the present study to developing a class of methods that can be adapted to the specifications of a galaxy-galaxy strong lensing population obtained with the next generation of optical, near-IR and radio telescopes. In particular, we confine ourselves to a setting where the main methodological points can be made without detailed modeling of the detector capabilities and the deliverable lensing dataset, which is outside of the scope of the current paper.

We now describe our models for the background source, lensing galaxy and population parameters of the lens systems used in this study. 

\subsection{Background source}

We model the background source galaxies using a S\'{e}rsic profile
\begin{equation}
I(r)=I_{e} \exp \left\{-b_{n}\left[\left(\frac{r}{r_{e}}\right)^{1 / n}-1\right]\right\},
\end{equation}
where $r_e$ is the effective circular half-light radius, $n$ is the S\'{e}rsic index, $b_n$ is a factor depending on $n$ that ensures that $r_e$ contains half the total intensity form the source galaxy, given by~\citep{1999A&A...352..447C}
\begin{equation}
b_n \approx 2 n - \frac{1}{3} + \frac{4}{405 n} + \frac{46}{25515 n^2} + \frac{131}{1148175 n^3} - \frac{2194697}{30690717750 n^4}.
\end{equation}

We use $n=1$ for the source galaxies, corresponding to an exponential profile and consistent with expectation for blue-type galaxies at the relevant redshifts. Treatment of the other parameters in the context of population studies is described in Secs.~\ref{sec:observations} and~\ref{sec:populations} below.

\subsection{Lensing host galaxy}

Cosmological $N$-body simulations suggest that the dark matter distribution in structures at galactic scales can be well-described by a universal, spherically symmetric NFW profile. However, strong lensing probes a region of the host galaxy much smaller than the typical virial radii of galaxy-scale dark matter halo, and the mass budget here is dominated by the baryonic bulge component of the galaxy. Taking this into account, the total mass budget of the lensing host galaxy, being early-type, can be well describe by a singular isothermal ellipsoid (SIE) profile, known as the bulge-halo conspiracy since neither the dark matter nor the baryonic components are individually isothermal. The host profile is thus described as
\begin{equation}
\rho(x, y)=\frac{\sigma_{v}^{2}}{2 \pi G\left(x^{2} / q+q y^{2}\right)}
\end{equation}
where $\sigma_{v}$ is the velocity dispersion of the lens galaxy $q$ is the ellipsoid axis ratio, with $q=1$ corresponding to a spherical profile. The Einstein radius for this profile, giving the characteristic lensing scale, is given by
\begin{equation}
\theta_{\mathrm{E}}=4 \pi\left(\frac{\sigma_{v}}{c}\right)^{2} \frac{D_{l s}\left(z_{l}, z_{s}\right)}{D_{s}\left(z_{s}\right)}
\end{equation}
where $D_{ls}$ and $D_s$ are respectively the angular diameter distances from the source to the lens planes and from the source plane to the observer respectively. 

The deflection field for this profile is given by...

Although the total galaxy mass (baryons + dark matter) describe the macro lensing field, for the purposes of describing substructure we require being able to map the SIE properties onto the properties of the host dark matter halo. To do this, we use... 

We described the population parameters we use to describe the host ellipticity and velocity dispersion (and thus its Einstein radius and halo mass) in Secs.~\ref{sec:observations} and~\ref{sec:populations} below.

% - Calculate halo mass from velocity dispersion using relation from that paper
% - Have a few ideas to write down subhalo mass function
% 	- Use unified model paper
% 	- Use CLUMPY paper
% 	- Use Despali paper (but this doesn't have the norm)
% - Assuming \propto scaling for host halo mass
% - Scale number of subhalos to our region using ratio of projected mass profiles?
% - Draw 3D distances and truncate
% - Actually, just distribute isotropically and use S-C concentration, no truncation

\subsection{Lensing substructure}

\subsection{Observational considerations}
\label{sec:observations}


\subsection{Population statistics of the lens sample}
\label{sec:populations}

We use the \texttt{LensPop} code~\citep{Collett:2015roa}.


\acknowledgements


\software{
}

\bibliographystyle{aasjournal}
\bibliography{lensing-lfi}

\end{document}
